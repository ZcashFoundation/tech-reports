\documentclass{article}

\usepackage{amsthm, amsfonts, amssymb, amsmath, graphicx, cancel, framed}

\usepackage[hyphens]{url}
\urlstyle{same}
\usepackage{xifthen}
\usepackage[usenames,svgnames,table]{xcolor}

\setlength{\marginparsep}{8pt}
\setlength{\marginparwidth}{40pt}
\setlength{\marginparpush}{5pt}
\newcounter{mn}
\setcounter{mn}{1}
\newcommand{\superscript}[1]{\ensuremath{{}^{\textrm{\scriptsize #1}}}}
\newcommand{\roughtext}[1]{\begin{color}{SkyBlue}#1\end{color}}
\newcommand{\mntext}[1]{\colorbox{SkyBlue}{\begin{color}{black}#1\end{color}}}
\newcommand{\mn}[2][]{{\tiny\superscript{\mntext{\arabic{mn}}}}\marginpar{\scriptsize{
  \ifthenelse{\isempty{#1}}
  {\mntext{\parbox{0.95\marginparwidth}{\superscript{\arabic{mn}}~\raggedright{#2}}}}
  {\mntext{\parbox{0.95\marginparwidth}{\superscript{\arabic{mn}}#1 says:~\raggedright{#2}}}}
}}\stepcounter{mn}}

\newcommand\supth{\ensuremath{^{\textrm{\scriptsize th}}}}



\title{Zcash Threat Model and Network Privacy Assessment}
\author{The Zcash Foundation}

\begin{document}

    \maketitle
\section{Introduction}

The amount of data leaked about individuals online is ever growing.
Financial data in particular
provides a highly granular lens about personal daily habits, but securing one's
own financial privacy is difficult due to the value of tracking financial
habits for advertising purposes. Fortunately, Zcash ensures strong financial
privacy using zero-knowledge proofs along with randomization
techniques~\cite{zcash-spec} to
protect against data leakage that could lead to deanonymization of a payer or
recipient of Zcash. This approach ensures that the raw bytes of a
Zcash \emph{shielded} transaction cannot leak identifiable information about the
payer, payee, or amount.

However, an adversary observing the network over which Zcash clients and nodes
send transactions could perform passive or active attacks with the goal of
linking
users and their end recipients, even if that adversary cannot decrypt the Zcash
shielded transaction directly. Consequently, it is important to assess the
how such adversaries could be successful, what information is required in order
to perform such deanonymization attacks, and what such adversaries require to
do so.

In this technical report, we consider the above questions, and present the
threat model for Zcash more formally, assuming the use of shielded
transactions. We identify what constitutes sensitive information in the Zcash
ecosystem, what adversaries may exist, and enumerate the attack vectors that
such adversaries may employ. From this model, we then consider several
different network privacy mechanisms, and analyze the extent to which these
mechanisms improve the privacy and security of Zcash users. Finally, we lay out
several near-term and long-term recommendations for improvements.

\textit{Organization}.
We begin in Section~\ref{background} by discussing background information
useful to understanding Zcash and its threat model. In
Section~\ref{threat-model}, we introduce a threat model for Zcash, by
discussing sensitive information that attackers could observe in the Zcash
network, reviewing possible attacker profiles and their associated
capabilities and powers, and finally reviewing a range of potential attack
vectors.

\mn[CK]{TODO add in the second half of this work after it is added}

\section{Background} \label{background}

\subsection{Zcash Shielded Transactions}
\label{shielded}

While Zcash does allow for use of unshieleded transactions, in this technical
report we focus entirely on the threat model assuming the use of shielded
transactions. However, we will now briefly describe
the difference between the two transaction types.
As mentioned before, unshielded transactions are similar
to Bitcoin transactions, and expose the pseudonyms of the payer, payee, and the
amount paid. When this information is exposed to the network and persisted
indefinitely to the Blockchain, it is effectively “Twitter for your bank
account”.

Shielded transactions, on the other hand, expose what is essentially a “one
time pad” to a network observer. Because not only the transaction information
is encrypted, the encryption process itself is randomized such that two
transactions that have identical payers, payees, and transaction amounts result
in bytes that are completely indistinguishable from randomly-generated bytes.
In other words, given only a series of shielded transactions, an adversary
would not be able to gain enough information to distinguish any information
about the transaction paintext.


\subsection{Comparison of Zcash Privacy Expectations to Bitcoin}

While Zcash provides the ability to make shielded transactions to completely
hide the information contained within a transaction, users of Zcash can also
make non-shielded transactions. Because Zcash is a fork of Bitcoin,
non-shielded transactions consequently have effectively the same expectation of
privacy as plain Bitcoin transactions, which had been proven to be insecure
against de-anonymization attacks~\cite{anon-bitcoin}. In this post, we’ll
evaluate the threat model for Zcash considering only shielded transactions.

While both Zcash and Bitcoin have future goals of
stable Tor integration and other network-privacy mechanisms, this routing via
an anonymity layer does not prevent attacks that examine the bytes of the
transaction itself. So unshielded transactions in Zcash and all Bitcoin
transactions leak information to a network observer, which can be exploited to
perform deanonymization attacks.

\section{Zcash Threat Model}
\label{threat-model}

\subsection{Analysis of Sensitive Information Exposure}
As described in Section~\ref{shielded} an adversary could gain more information
about a shielded transaction by observing information exposed to the network.
Keeping this in mind, and assuming that a transaction is shielded, we now
review sensitive information that could be exposed and used for malicious
purposes by a adversary.
We divide these information leaks into \emph{in scope} to the Zcash threat
model, and \emph{out of scope}.

\subsubsection{In Scope}

Note that an adversary has the ability to observe both \emph{on-chain} visible
data as well as information or behaviour that is exposed to the network during
the process of submitting a new transaction or receiving a transaction.

\textbf{Linking (sender identity, transaction, receiver identity)}:
Specifically, this three-tuple of sender identity, transaction, and receiver
identity allows for an adversary to link the fact that a
specific sender of Zcash made a payment to a specific receiver, even though the
amount of the payment is not disclosed. While the sender and receiver identity
is not exposed by the Zcash transaction itself, an adversary
could gain this information by observing the network or colluding
with one of the involved parties. For example, if the recipient is
malicious and colludes with a party that can link the sender to their
transaction, or if the recipient leaks to a provider which transaction they
are interested in, the sender and receiver could be linked to a particular
transaction.


\textbf{Unique fingerprints of user habits}: By observing behavior of Zcash
transactions sent over the network, an adversary could start to build a
“fingerprint” of a user’s behavior, partially when using machine learning
algorithms to classify traffic. Information that is exposed in this
category includes the time of day that a transaction occurs, the frequency
of transactions, and even the route that a transaction passes through from
sender to receiver.

Further, ``on-chain'' information could also allow for fingerprintability. For
example, fees and timing of transactions could possibly provide additional
information to an observer.

\subsubsection{Out of Scope}

\textbf{Learning the tuple (sender or receiver identity, transaction)}:
Note that a network observer can also gain a subset of sender/receiver
linkability information by
observing just a sender \emph{or} receiver's identity linked to a specific
transaction. We consider this use case to be out of scope to the threat model
of Zcash, as such
information---to the best of our assessment---simply leaks that the sender or
receiver is participating in sending and receiving Zcash, without any further
details. Consequently, we consider such an information leak to be out of scope.

\subsection{Adversarial Model}

We now review possible adversaries that may be motivated to compromise Zcash users'
privacy.

\textbf{Regular Zcash user}. Allowed to send and receive transactions and act
outside the protocol, just as a real user.

\textbf{Regular Zcash Node Operator}. Can operate one (or many) full Zcash
nodes, and can store and forward all traffic that is sent through the node. Can
query for network information, and store and examine all information it
receives.

\textbf{DNS seeder operators}. Operates the node that is used by new nodes when
bootstrapping to the Zcash network, in order to learn about other nodes in the
network to begin communicating with them directly.

\textbf{Internet Service Providers (ISPs)}. Can view traffic sent and received
from either users or Zcash nodes. Can store observed traffic, forward traffic
to other parties, and compare traffic with other traffic.

\textbf{Government actors}. Can issue secret subpoenas and force ISPs and
regular Zcash users to take actions they may not wish to take, such as turning
over secret keys or server logs.

\subsection{Attack Vectors}

We now review known attacks in the literature that have been described for
decentralized systems similar to Zcash, although not all attacks have been
demonstrated against Zcash specifically. Notably, these attacks leverage
decentralized networks where information about the network may or may not be
consistent.

\textbf{Epistemic attacks}. A network observer could perform an epistemic
attack by observing unique routing information that allows for eventual
deanonymization of that user. Again, such attacks are possible in Zcash because
users do not control routing information for their transaction.

One example of epistemic attacks against cryptocurrency networks involve the
probability of ``super-connected'' nodes that can link the node from which a
transaction originated~\cite{dandelion, fanti2018dandelion}.

\textbf{Fingerprinting attacks}. Fingerprinting user behavior when making a
transaction can lead to deanonymizing that user, even if shielded transactions
are used. Fingerprinting can be performed by a range of adversaries across many
different settings. For example, a malicious light wallet node could observe
the frequency and timing of a user’s transactions, or even the recipient of a
user’s transaction could create a profile of the person making payments to them
via timing and frequency of their payments, even if that user wishes to remain
anonymous.

\textbf{Denial of service attacks}. Such attacks could be performed against
Zcash nodes, such as DNS seeders refusing to respond to certain queries, or
against Zcash users, such as light wallets refusing to service certain classes
of IP addresses.

\textbf{Partitioning attacks}. The Zcash network itself could be partitioned,
such that some nodes think they are aware of the entire network, but only
instead be aware of a small subset of nodes. Such attacks could be performed by
DNS seeders (again, by lying about the state of the network), or even by
highly-connected nodes. Such attacks are well-known in the literature and in
practice and demonstrated to be practical.

\textbf{End to end correlation attacks}. This attack could occur by an
adversary that can obtain the (sender identity, transaction, receiver identity)
tuple discussed in Section~\ref{threat-model}. Such an attack is possible
only when the recipient uses a light wallet in such a manner that leaks the
transaction that the recipient is interested in, as further described in
Section~\cite{light-wallet-spec}. Otherwise, as all recipients download information about every
transaction via a broadcast-like protocol, and hence an adversary cannot gain a
true end-to-end view, unless the
recipient themselves is compromised or malicious.







\bibliographystyle{plain}
\bibliography{threat-model-network-privacy}

\end{document}
