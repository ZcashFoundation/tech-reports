\documentclass{article}

\usepackage{amsthm, amsfonts, amssymb, amsmath, graphicx, cancel, framed}

\usepackage[hyphens]{url}
\urlstyle{same}


\title{Zcash Threat Model and Network Privacy Assessment}
\author{The Zcash Foundation}

\begin{document}

    \maketitle
\section{Introduction}

In these days where too much of our lines are online, the amount of data leaked
about our personal lives is growing ever greater. Financial data in particular
provides a highly granular lens about one’s personal life, but obtaining
financial privacy is difficult to obtain due to the value of tracking financial
habits for advertising purposes. Fortunately, Zcash ensures strong financial
privacy using zero-knowledge proofs (as well as randomization techniques) to
protect against data leakage. This approach ensures that the raw bytes of a
Zcash shielded transaction cannot leak identifiable information about the
payer, payee, or amount.

However, an adversary observing the network over which Zcash clients and nodes
send transactions  can perform passive or active attacks resulting in linking
users and their end recipients, even if that adversary cannot decrypt the Zcash
shielded transaction directly. Consequently, it is important for us as Zcash
protocol designers to ensure that such adversaries are considered within our
threat model, and to employ best practices in network privacy to mitigate such
attacks.

The purpose of this post is to sketch out Zcash’s existing threat model,
assuming the use of shielded transactions. To do this, we will identify 1) what
is considered sensitive information in Zcash, 2) where this information is
exposed, 3) what adversaries may exist, and finally 4) enumerate to the best of
our ability the attacks that these adversaries can perform. In our next post,
we will use this analysis to evaluate a range of network privacy mechanisms,
and identify immediate next steps that we can take to ensure the most
significant privacy improvements.

\textit{Organization}.
TODO

\section{Background}

\subsection{Zcash}

TODO

\subsection{Comparison of Zcash Privacy Expectations to Bitcoin}

While Zcash provides the ability to make shielded transactions to completely
hide the information contained within a transaction, users of Zcash can also
make non-shielded transactions. Because Zcash is a fork of Bitcoin,
non-shielded transactions consequently have effectively the same expectation of
privacy as plain Bitcoin transactions, which is not much. In this post, we’ll
evaluate the threat model for Zcash considering only shielded transactions.

While both Zcash and Bitcoin have future goals of
stable Tor integration and other network-privacy mechanisms, this routing via
an anonymity layer does not prevent attacks that examine the bytes of the
transaction itself. So unshielded transactions in Zcash and all Bitcoin
transactions leak information to a network observer, which can be exploited to
perform deanonymization attacks.

\section{Zcash Threat Model}

\subsection{Analysis of Sensitive Information Exposure}
TODO

\subsection{Adversarial Model}
TODO

\subsection{Attack Vectors}

\bibliographystyle{plain}
\bibliography{threat-model-network-privacy}

\end{document}
