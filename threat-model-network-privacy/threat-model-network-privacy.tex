\documentclass{article}

\usepackage{amsthm, amsfonts, amssymb, amsmath, graphicx, cancel, framed}
\usepackage{mathtools}
\usepackage{multicol}
\usepackage{multirow}
\usepackage{amsmath}
\usepackage{amsfonts, txfonts}
\usepackage{ifsym}

\usepackage{commath}
\usepackage{amsthm}
\usepackage{caption}
\captionsetup[figure]{font=small,belowskip=-10pt}
\captionsetup[figure]{font=small,aboveskip=3pt}
\usepackage{wasysym}

\usepackage[hyphens]{url}
\urlstyle{same}
\usepackage{xifthen}
\usepackage[usenames,svgnames,table]{xcolor}

\setlength{\marginparsep}{8pt}
\setlength{\marginparwidth}{40pt}
\setlength{\marginparpush}{5pt}
\newcounter{mn}
\setcounter{mn}{1}
\newcommand{\superscript}[1]{\ensuremath{{}^{\textrm{\scriptsize #1}}}}
\newcommand{\roughtext}[1]{\begin{color}{SkyBlue}#1\end{color}}
\newcommand{\mntext}[1]{\colorbox{SkyBlue}{\begin{color}{black}#1\end{color}}}
\newcommand{\mn}[2][]{{\tiny\superscript{\mntext{\arabic{mn}}}}\marginpar{\scriptsize{
  \ifthenelse{\isempty{#1}}
  {\mntext{\parbox{0.95\marginparwidth}{\superscript{\arabic{mn}}~\raggedright{#2}}}}
  {\mntext{\parbox{0.95\marginparwidth}{\superscript{\arabic{mn}}#1 says:~\raggedright{#2}}}}
}}\stepcounter{mn}}

\newcommand\supth{\ensuremath{^{\textrm{\scriptsize th}}}}



\title{Zcash Threat Model and Network Privacy Assessment}
\author{The Zcash Foundation}

\begin{document}

    \maketitle
\section{Introduction}

The amount of data leaked about individuals online is ever growing.
Financial data in particular
provides a highly granular lens about personal daily habits, but securing one's
own financial privacy is difficult due to the value of tracking financial
habits for advertising purposes. Fortunately, Zcash ensures strong financial
privacy using zero-knowledge proofs along with randomization
techniques~\cite{zcash-spec} to
protect against data leakage that could lead to deanonymization of a payer or
recipient of Zcash. This approach ensures that the raw bytes of a
Zcash \emph{shielded} transaction cannot leak identifiable information about the
payer, payee, or amount.

However, an adversary observing the network over which Zcash clients and nodes
send transactions could perform passive or active attacks with the goal of
linking
users and their end recipients, even if that adversary cannot decrypt the Zcash
shielded transaction directly. Consequently, it is important to assess the
how such adversaries could be successful, what information is required in order
to perform such deanonymization attacks, and what such adversaries require to
do so.

In this technical report, we consider the above questions, and present the
threat model for Zcash more formally, assuming the use of shielded
transactions. We identify what constitutes sensitive information in the Zcash
ecosystem, what adversaries may exist, and enumerate the attack vectors that
such adversaries may employ. From this model, we then consider several
different network privacy mechanisms, and analyze the extent to which these
mechanisms improve the privacy and security of Zcash users. Finally, we lay out
several near-term and long-term recommendations for improvements.

\textit{Organization}.
We begin in Section~\ref{background} by discussing background information
useful to understanding Zcash and its threat model. In
Section~\ref{threat-model}, we introduce a threat model for Zcash, by
discussing sensitive information that attackers could observe in the Zcash
network, reviewing possible attacker profiles and their associated
capabilities and powers, and finally reviewing a range of potential attack
vectors.

\section{Background} \label{background}

\subsection{Zcash Shielded Transactions}
\label{shielded}

While Zcash does allow for use of unshieleded transactions, in this technical
report we focus entirely on the threat model assuming the use of shielded
transactions. However, we will now briefly describe
the difference between the two transaction types.
As mentioned before, unshielded transactions are similar
to Bitcoin transactions, and expose the pseudonyms of the payer, payee, and the
amount paid. When this information is exposed to the network and persisted
indefinitely to the Blockchain, it is effectively “Twitter for your bank
account”.

Shielded transactions, on the other hand, expose what is essentially a “one
time pad” to a network observer. Because not only the transaction information
is encrypted, the encryption process itself is randomized such that two
transactions that have identical payers, payees, and transaction amounts result
in bytes that are completely indistinguishable from randomly-generated bytes.
In other words, given only a series of shielded transactions, an adversary
would not be able to gain enough information to distinguish any information
about the transaction plaintext.


\subsection{Comparison of Zcash Privacy Expectations to Bitcoin}

While Zcash provides the ability to make shielded transactions to completely
hide the information contained within a transaction, users of Zcash can also
make non-shielded transactions. Because Zcash is a fork of Bitcoin,
non-shielded transactions consequently have effectively the same expectation of
privacy as plain Bitcoin transactions, which had been proven to be insecure
against de-anonymization attacks~\cite{anon-bitcoin}. In this post, we’ll
evaluate the threat model for Zcash considering only shielded transactions.

While both Zcash and Bitcoin have future goals of
stable Tor integration and other network-privacy mechanisms, this routing via
an anonymity layer does not prevent attacks that examine the bytes of the
transaction itself. So unshielded transactions in Zcash and all Bitcoin
transactions leak information to a network observer, which can be exploited to
perform deanonymization attacks.

\subsection{Zcash Protocol for Producing and Receiving Transactions}

\textbf{Producing a Transaction (Spending Funds)}.
A user wishing to spend
funds can publish a new transaction to the network in several ways. The user
could operate a \emph{full node}, meaning this node also participates in all
network behaviour, such as gossiping transactions, responding to queries for
the current state of the network, etc.

The user can also spend funds via a \emph{light
client} by publishing their transaction to a
\emph{light wallet node}, which itself is a full node which additionally can handle
light wallet functionality. Note that while publishing a transaction to a light
wallet, the user exposes the fact that they are making a Zcash transaction (as
the light wallet learns a particular transaction, along with the IP address of
the user). However, the light wallet will not learn the identity of the
receiver of the funds~\cite{light-wallet-spec}.

\textbf{Receiving a Transaction (Accepting Funds)}. A user receiving a
transaction similarly has two options for how to receive these funds, either
via a full node or via a light wallet node.

An important point to emphasize that because Zcash is a broadcast protocol (all
full nodes sync the full state of the network), a user receiving transactions
via a full node will not expose which transaction they are interested.

Similarly, a user fetching transactions from a light wallet will also fetch all
block headers and therefore not expose which transaction they are interested
in. However, one slight exception exists for clients that wish
to learn the full details of a transaction (such as the memo field). For this
case,
they must query the light wallet for those details separately. As such, the
light wallet can link a recipient to a particular transaction in the case the
user queries for extended transaction fields for a specific block~\cite{light-wallet-spec}.

\section{Zcash Threat Model}
\label{threat-model}

\subsection{Analysis of Sensitive Information Exposure}
As described in Section~\ref{shielded} an adversary could gain more information
about a shielded transaction by observing information exposed to the network.
Keeping this in mind, and assuming that a transaction is shielded, we now
review sensitive information that could be exposed and used for malicious
purposes by a adversary.
We divide these information leaks into \emph{in scope} to the Zcash threat
model, and \emph{out of scope}.

\subsubsection{In Scope}
\label{in-scope}

Note that an adversary has the ability to observe both \emph{on-chain} visible
data as well as information or behaviour that is exposed to the network during
the process of submitting a new transaction or receiving a transaction.

\textbf{Linking (sender identity, transaction, receiver identity)}:
Specifically, this three-tuple of sender identity, transaction, and receiver
identity allows for an adversary to link the fact that a
specific sender of Zcash made a payment to a specific receiver, even though the
amount of the payment is not disclosed. While the sender and receiver identity
is not exposed by the Zcash transaction itself, an adversary
could gain this information by observing the network or colluding
with one of the involved parties. For example, if the recipient is
malicious and colludes with a party that can link the sender to their
transaction, or if the recipient leaks to a provider which transaction they
are interested in, the sender and receiver could be linked to a particular
transaction.


\textbf{Unique fingerprints of user habits}: By observing behavior of Zcash
transactions sent over the network, an adversary could start to build a
“fingerprint” of a user’s behavior, partially when using machine learning
algorithms to classify traffic. Information that is exposed in this
category includes the time of day that a transaction occurs, the frequency
of transactions, and even the route that a transaction passes through from
sender to receiver.

Further, ``on-chain'' information could also allow for fingerprintability. For
example, fees and timing of transactions could possibly provide additional
information to an observer.

\subsubsection{Out of Scope}
\label{out-of-scope}

\textbf{Learning the tuple (sender or receiver identity, transaction)}:
Note that a network observer can also gain a subset of sender/receiver
linkability information by
observing just a sender \emph{or} receiver's identity linked to a specific
transaction. We consider this use case to be out of scope to the threat model
of Zcash, as such
information---to the best of our assessment---simply leaks that the sender or
receiver is participating in sending and receiving Zcash, without any further
details. Consequently, we consider such an information leak to be out of scope.

\textbf{Unforeseen Software Flaws}: While our team developing Zcash software
works diligently to prevent software bugs and vulnerabilities to the best of
our ability, we do not
consider such flaws in scope to our threat model, as such cases are unforeseen
and we cannot rule them out completely. As such, when we become aware of
vulnerabilities or flaws, we will fix them, but we cannot claim such
occurrences will not occur in the future.

\textbf{User Behavior Outside of the Zcash Protocol}: Zcash is not used in a
vacuum; spending and receiving Zcash is linked to real-world value. As such, we
do not consider user behavior outside of the Zcash protocol to be in scope to
our threat model, even if that behavior results in spending or receiving Zcash.
For example, someone who goes to a website and wishes to purchase an item in
Zcash may be unlinkable purely when observing the Zcash network, but their
behavior when browsing the website can still be observable to a network
attacker. In order to spend Zcash in a \emph{truly} private way, the user must
use another privacy-preserving technology to hide their spending behavior
outside of Zcash, such as accessing that website over Tor.

\textbf{Block Access Patterns}: Even if a user accessed
information about the Zcash blockchain using an anonymity tool such as Tor, the
access patterns for specific blocks could leak information to network
adversaries, such as the time of day that users requested information about
that block, or its popularity. However, preventing such information leaks
can likely only be solved using a full broadcast system, which is already
an option for full Zcash notes. For optimizations, we do not consider the
access patterns on blocks to be in scope to the Zcash threat model.

\textbf{Malicious Senders or Receivers of Zcash}: We consider senders or
receivers of Zcash which act honestly within the protocol but maliciously
outside of the protocol (such as creating a ``honeypot'' for which Zcash users
are tricked into sending valid transactions to) to be out of scope for our
threat model.

\subsection{Applicable Only for Unshielded Transactions.}

\textbf{Epistemic attacks}. A network observer could perform an epistemic
attack by observing unique routing information that allows for eventual
deanonymization of that user. Again, such attacks are possible in Zcash because
users do not control routing information for their transaction.

One example of epistemic attacks against cryptocurrency networks involve the
probability of ``super-connected'' nodes that can link the node from which a
transaction originated~\cite{dandelion, fanti2018dandelion}.

However, let's consider the case when only shielded transactions are used. In
such a setting, the recipient of the transaction will not be disclosed if they
are using a full node, and only disclosed if they fetch additional parameters
if using a light wallet (such as by fetching the Zcash memo field). As such,
since our analysis only considers the case where shielded transactions are
used, we deem epistemic attacks such as those presented by the Dandelion
authors as not applicable for Zcash.

\textbf{Partitioning attacks}. The Zcash network itself could be partitioned,
such that some nodes think they are aware of the entire network, but only
instead be aware of a small subset of nodes. Such attacks could be performed by
DNS seeders (again, by lying about the state of the network), or even by
highly-connected nodes. Such attacks are well-known in the literature and in
practice and demonstrated to be practical.

However, when considering deanonymization attacks against Zcash users, if
shielded transactions are used, the same threat model holds for users
regardless of whether an attacker can successfully perform a partitioning
attack on the network.


\subsection{Adversarial Model}

We now review possible adversaries that may be motivated to compromise Zcash users'
privacy.

\textbf{Regular Zcash user}. Allowed to send and receive transactions and act
outside the protocol, just as a real user.

\textbf{Regular Zcash Node Operator}. Can operate one (or many) full Zcash
nodes, and can store and forward all traffic that is sent through the node. Can
query for network information, and store and examine all information it
receives.

\textbf{DNS seeder operators}. Operates the node that is used by new nodes when
bootstrapping to the Zcash network, in order to learn about other nodes in the
network to begin communicating with them directly.

\textbf{Internet Service Providers (ISPs)}. Can view traffic sent and received
from either users or Zcash nodes. Can store observed traffic, forward traffic
to other parties, and compare traffic with other traffic.

\textbf{Government actors}. Can issue secret subpoenas and force ISPs and
regular Zcash users to take actions they may not wish to take, such as turning
over secret keys or server logs.

\subsection{Attack Vectors}
\label{attack-vectors}

We now review known attacks in the literature that have been described for
decentralized systems that are applicable to Zcash even when shielded addresses
are used. Notably, some these attacks leverage
decentralized networks where information about the network may or may not be
consistent, or require the use of light wallets in a setting where user
behavior can be differentiated.

\textbf{Fingerprinting attacks by Observing Timing of Transactions}.
Fingerprinting timing of when transactions are made
can lead to deanonymizing the sender and
receiver \emph{only} in the light wallet setting, even if shielded transactions
are used. We note that in the full node setting, such attacks are not possible
due to the full broadcast nature once transactions are published to the Zcash
blockchain.

However, timing-based fingerprinting can be performed by a
malicious light wallet node or even a malicious or compromised recipient. In
this setting, these parties could build a profile of the person making payments
to them by observing the timing of their payments, even if the sender wishes to remain
anonymous.

While this attack can be amplified when information external to Zcash, such as
if a user visits a website and exposes their IP address, and then makes a Zcash
transaction, possibly allowing the recipient to link the two events. However,
we consider this event to be out of scope, as further discussed in
Section~\ref{out-of-scope}.

\textbf{Fingerprinting attacks by Observing Behavior of Transactions}.
Fingerprinting user behavior such as the frequency and number of transactions
made can lead to deanonymizing senders and receivers in the setting where light
wallets are used, even the transaction is shielded.
For example, a malicious light wallet node could observe
the frequency and pattern of a user’s transactions, and build a pattern to
match against over time.

\textbf{Denial of service attacks}. Such attacks could be performed against
Zcash nodes, such as DNS seeders refusing to respond to certain queries, or
against Zcash users, such as light wallets refusing to service certain classes
of IP addresses.

\textbf{On-Path end to end correlation attacks}. This attack could occur by an
adversary that can obtain the (sender identity, transaction, receiver identity)
tuple discussed in Section~\ref{threat-model}. By \emph{on-path end to end
correlation}, we include attacks performed by nodes that are directly on the
path between a sender and receiver when publishing or receiving a transaction.
In the current model of Zcash, such an attacks are practical
only when the recipient uses a light wallet in such a manner that leaks the
transaction that the sender or receiver is publishing, as further described in
Section~\cite{light-wallet-spec}.


\section{Review of Network Privacy Approaches}
\label{network-privacy-review}

Ideally, some of the attacks described in Section~\ref{attack-vectors} could be
mitigated by using a network privacy layer. We now review three
classes of network privacy approaches, and then in
Section~\ref{network-privacy-assessment}, determine how these approaches
address the existing described attacks against Zcash.

For brevity, we only review
Dandelion~\cite{Fanti:2018:DLC,BojjaVenkatakrishnan:2017:DRB},
Tor~\cite{tor-specification}, and
Loopix-based mixnets~\cite{Piotrowska:2017:LAS}.
Further, we review private information retrieval (PIR) as a method which can be
used in conjunction with the above systems.

\textbf{Dandelion.}
Dandelion~\cite{BojjaVenkatakrishnan:2017:DRB} and
Dandelion++~\cite{Fanti:2018:DLC} is a lightweight gossip protocol aimed at
adding additional network privacy for
distributed networks such as cryptocurrencies. Dandelion protects against
\emph{passive}
deanonymization attacks, but does not consider active or targeted attacks. Such
passive deanonymization attacks could be conducted by a ``super node'' that has
a high degree of connections to other nodes (and could either be a single node
or a botnet where adversarial machines share information). As such, it is
assumed that this adversary is honest-but-curious, following the gossip protocol
but wishes to learn as much information about users as it can directly observe.
However, Dandelion++ does consider a stronger adversarial model where nodes are
allowed an arbitrarily-number of connections to other nodes (acting outside of the
Bitcoin gossip spec which only allows 8).

Both Dandelion variants follow a randomized design for how transactions are
gossiped to the network, differing from the Bitcoin design where nodes publish
transactions as widely as possible as quickly as possible. In the Dandelion
design, whenever a node receives a transaction from a neighbor, it
first flips a coin to determine if the traffic is sent to a single
neighbor (constituting the ``stem'' phase) or to all the node's connections
(constituting the ``fluff'' phase). Such an approach frustrates the ability for
a super node to link a specific transaction to the node which originally
published that transaction.


\textbf{Tor.}
Tor~\cite{tor-specification} is an anonymity network that today has over 2.5
million users and a network
size of over 6,500 nodes. Tor supports applications that require low latency,
such as browsing the Internet anonymously or streaming videos. In order to
support such a use case, Tor assumes that it is hard for network adversaries to
gain an end-to-end view and provide correlation attacks, such as by injecting
timing or dropping packets to test if the traffic it can view entering the
network is the same as the traffic it can view leaving the network.

Tor distributes its routing information (i.e, information about each relay) via
an authenticated document called the \emph{consensus}, which is signed by a
threshold number of trusted servers called \emph{directory authorities}. In
doing so, Tor ensures that all clients and relays have a consistent view of the
network. By distributing a global authenticated document to all network
participants, Tor avoids epistemic or routing attacks unlike completely
decentralized networks which cannot guarantee a user or relay's view of the
network is authentic or that all users fall within a global anonymity set.


\textbf{Loopix-based Mixnets.}
While a range of mix network designs have been introduced in the literature, in
this assessment we consider only those which instantiate the
Loopix~\cite{Piotrowska:2017:LAS} design, which provides improved latency
guarantees over prior designs. Similar to other mix network designs, Loopix
uses dummy packets and message delays in order to protect against adversaries
performing fingerprinting and end-to-end attacks of user traffic.

While Loopix specifies how messages are routed through a network, Loopix-based
networks still require safely distributing network information to users and
nodes. As one example, Katzenpost~\cite{katzenpost}, a mix network which implemented
Loopix,  used a similar model to Tor
by leveraging trusted network authorities to sign and distribute the state of
the network. However, alternative network distribution mechanisms can be used,
but similarly may be subject to epistemic and path-routing attacks similar to
other distributed networks.

\textbf{Private Information Retrieval (PIR).}
Even though the above network anonymity systems disassociate a sender or
receiver's identity from the transaction, nodes can still observe which blocks
are being fetched and perform fingerprinting attacks using this information.
One mechanism that can be used in conjunction with network anonymity tools is
private information retrieval (PIR)~\cite{pir}, which allows users to query information
while preventing the service that hosts this information from learning the
query.


\section{Assessment of Network Privacy Approaches to Zcash Privacy}
\label{network-privacy-assessment}

We now review the extent to which Dandelion++, Tor, and a Loopix-based mixnet
protect against the existing network-based attacks against Zcash described in
Section~\ref{attack-vectors}.

Note that we assume an attacker does not control either the sender or receiver
of funds. We assume the mixnet only delays transactions and does not generate
dummy packets (as generating dummy transactions that are verifiable require
higher-level application support, and cannot be a property entirely provided by
the mixnet itself).

We assume the mixnet is a \emph{seperate} anonymity network entirely, such as
in the style of Nym, and is \emph{not} part of the cryptocurrency network
itself (meaning that the initiator of a transaction forwards this transaction
through the mixnet before it is published to the cryptocurrency network).

\begin{table}[t]
  \caption{Effectiveness of network privacy mechanisms for Zcash
  security and privacy}
  \label{network-zcash-assessment}

\footnotesize

  \CIRCLE=protects against; \Circle=does not protect against;
  $\bigstar$=Not a threat in this setting;

  Mixnet=Loopix-based routing capable of delaying packets but not creating
  dummy transactions;

  \medskip

  \begin{tabular}{ p{4.5em}| c | c | c | c | c | c | c}
    & & & & & \multicolumn{3}{c}{PIR}  \\
    & Attack & Dandelion++ & Tor & Mixnet  & Only PIR & Tor & Mixnet \\
 \hline
    \multirow{4}{*}{Full Node}  & Timing Fingerprinting &
    $\bigstar$ &   $\bigstar$ & $\bigstar$ & $\bigstar$ & $\bigstar$ & $\bigstar$ \\
    & Behavior Fingerprinting & $\bigstar$ & $\bigstar$ & $\bigstar$ &
    $\bigstar$ & $\bigstar$  & $\bigstar$ \\
    & Denial of Service & \Circle & \Circle & \Circle & \Circle & \Circle & \Circle \\
    & On-Path End to End Correlation & $\bigstar$ & $\bigstar$ & $\bigstar$ & $\bigstar$ & $\bigstar$ & $\bigstar$ \\

    \hline

    \multirow{4}{*}{Light Wallet} & Timing Fingerprinting & \Circle & \Circle & \CIRCLE & \Circle & \Circle & \CIRCLE \\
    & Behavior Fingerprinting & \Circle & \Circle & \Circle & \Circle & \CIRCLE & \CIRCLE \\
    & Denial of Service & \Circle & \Circle & \Circle & \Circle &  \Circle &
    \Circle \\
    & On-Path End to End Correlation & \Circle & \CIRCLE & \CIRCLE & \Circle &
    \CIRCLE & \CIRCLE \\

\end{tabular}
\end{table}

We summarize our results in Table~\ref{network-zcash-assessment}, but describe
our findings here.

\subsection{Analysis for Use of Full Nodes}

\textbf{Dandelion++.}
In the full node setting, Dandelion++ does not meaningfully change privacy
guarantees for users, as Dandelion++ only provides privacy in the setting of a
super node with knowledge of the network graph. Because receivers of
Zcash transactions operate in a full broadcast setting when using a full node,
Dandelion++ does not add meaningful additional privacy guarantees against
end-to-end correlation attacks. Further, the protocol does not add any
protection against fingerprinting or denial of service attacks.

\textbf{Tor.}
Again, in the setting where full nodes are used, the receiver cannot be linked
to a specific transaction, unless the receiver themselves is malicious. As
such, while Tor can hide which transaction a user made, in this threat model
that does not meaningfully change user privacy. Further, because Tor is a
low-latency network, use of Tor will not hide fingerprintable information.

\textbf{Mixnets.}
While the use of a Loopix-based mixnet raises the cost to an adversary to
conduct fingerprinting attacks that could result in end-to-end linking of
senders and receivers, this protection does not meaningfully increase user
security in the full node setting. Again, because receivers operate in a
full-broadcast mode, end-to-end linking of senders and receivers is not
possible when full nodes are used, unless the receiver themselves is
compromised or malicious.

As such, assuming receivers are honest, then mixnets do not meaningfully provide
additional privacy or security to Zcash users over Tor when operating in the
full node setting.

\textbf{Only PIR.}
Does not protect against sender information leakage, but as the receiver
receives in full broadcast, none of the above attacks apply.

\textbf{Tor + PIR.}
Because receivers of Zcash when using a full node operate in full broadcast
receiver mode---meaning that the user fetches the complete state of the network
and consequently does not leak which transaction they are interested in---use
of PIR in this setting does not meaningfully add privacy or security
protections.

As such, use of Tor + PIR in the setting where users operate with full nodes is
the same as the use of Tor.

\textbf{Mixnet + PIR.}
Again, because users operate with full nodes, even though the sender's identity
is hidden and so is any timing leaks, this does not add meaningful security.

\subsection{Analysis for Use of Light Wallets}

\textbf{Dandelion++.}
While useful in a non-shielded context, Dandelion++ does not protect against
end-to-end correlation attacks in the setting where light wallets are used.
Specifically, while Dandelion++ protects against super nodes that can observe
in-network gossip messages, Dandelion++ does not provide ``last-mile'' privacy,
meaning that users can still be deanonymized by the light wallets they use.

\textbf{Tor.}
Because receivers can be distinguished from other in a light wallet setting
when they fetch additional transaction details (because these users only fetch
additional details for the transaction they are specifically interested in),
adversaries can leverage timing and behavior-based fingerprinting attacks in
the light wallet setting, even when Tor is used. Because Tor is a low-latency
network, even if the receiver IP address is hidden from the light wallet, the
light wallet or a network observer can still observe the user's timing and
behavior when fetching transaction details.

\textbf{Mixnets.}
In the setting for light wallets, mixnets can prevent timing-based attacks, as
packets are delayed when forwarded through a mixnet, and consequently make
performing such attacks significantly more difficult for an adversary.

However, we note that mixnets in this setting cannot hide sender behavior, as
doing so would require creating ``dummy'' transactions, which becomes
complicated when doing so may require demonstrating adequate funds, etc.
Mixnets can hide receiver behavior to an external network observer by injecting
dummy packets, but cannot hide receiver behavior to the light wallet itself. As
such, we grant a half circle for the capability of mixnets to prevent behavior
based fingerprinting.

\textbf{Only PIR.}
We do not grant this property because even though the transaction that the receiver
fetches is hidden from the light wallet, the light wallet can learn the
receiver's identity as well as observe behavior patterns such as frequency of
access.

\textbf{Tor + PIR.}
PIR in the setting of Zcash ensures that a receiver can hide which transaction
they are interested in learning. Note that the use of PIR is helpful only in
the setting where receivers use light wallets and are interested in fetching
additional transaction details such as the memo field.

We grant a full circle because the light wallet does not learn
the sender and receiver identity nor the transaction that is being fetched.
Note that the light wallet can still observe information such as frequency of
access, but it is unclear how the light wallet can meaningfully use only this
information.

\textbf{Mixnet + PIR.}
Similarly to Tor, a mixnet used in conjunction with PIR protects against the
light wallet from learning both the identity of the sender and/or receiver. The
mixnet can discourage timing attacks but as it will not create dummy
transactions on behalf of the user, cannot protect against analysis such as
determining frequency of transactions. We deem the protection offered by
mixnets used with PIR to be slightly stronger than Tor + PIR because of timing
unlinkability, but all other protections in this model remain the same.

\bibliographystyle{plain}
\bibliography{threat-model-network-privacy}

\end{document}
